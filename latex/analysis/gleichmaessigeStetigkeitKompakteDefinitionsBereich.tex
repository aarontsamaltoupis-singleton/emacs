\documentclass[a4paper,oneside,11pt]{article}
\usepackage[utf8]{inputenc}

\usepackage[ngerman]{babel} % For correct hyphenation

\usepackage{mathtools}
\usepackage{amssymb,amsmath,amsthm}
\usepackage[mathscr]{eucal}

\usepackage[textwidth=17cm,top=1.5cm,bottom=1.5cm,nohead]{geometry}

\setlength{\parindent}{0mm} % no paragraph indentation


%%%%%%%%%%%%%%%%%%%%%%%%%%%%%%%%%%%%%%%%%%%%%%%%%%%%%%%

% Abbreviations

%%%%%%%%%%%%%%%%%%%%%%%%%%%%%%%%%%%%%%%%%%%%%%%%%%%%%%%
% single letters in different fonts 

%%%%%%%%%%%% mathematical bold  %%%%%%%%%%%%%%%%%%%%

\newcommand{\bA}{\mathbb{A}}
\newcommand{\bB}{\mathbb{B}}
\newcommand{\bC}{\mathbb{C}}
\newcommand{\bD}{\mathbb{D}}
\newcommand{\bE}{\mathbb{E}}
\newcommand{\bF}{\mathbb{F}}
\newcommand{\bG}{\mathbb{G}}
\newcommand{\bH}{\mathbb{H}}
\newcommand{\bI}{\mathbb{I}}
\newcommand{\bJ}{\mathbb{J}}
\newcommand{\bK}{\mathbb{K}}
\newcommand{\bL}{\mathbb{L}}
\newcommand{\bM}{\mathbb{M}}
\newcommand{\bN}{\mathbb{N}}
\newcommand{\bO}{\mathbb{O}}
\newcommand{\bP}{\mathbb{P}}
\newcommand{\bQ}{\mathbb{Q}}
\newcommand{\bR}{\mathbb{R}}
\newcommand{\bS}{\mathbb{S}}
\newcommand{\bT}{\mathbb{T}}
\newcommand{\bU}{\mathbb{U}}
\newcommand{\bV}{\mathbb{V}}
\newcommand{\bW}{\mathbb{W}}
\newcommand{\bX}{\mathbb{X}}
\newcommand{\bY}{\mathbb{Y}}
\newcommand{\bZ}{\mathbb{Z}}


%%%%%%%%% calligraphic %%%%%%%%%%%%%%%%%%%%%%%
\newcommand{\mc}[1]{\mathcal{#1}}

\newcommand{\cA}{\mathcal{A}}
\newcommand{\cB}{\mathcal{B}}
\newcommand{\cC}{\mathcal{C}}
\newcommand{\cD}{\mathcal{D}}
\newcommand{\cE}{\mathcal{E}}
\newcommand{\cF}{\mathcal{F}}
\newcommand{\cG}{\mathcal{G}}
\newcommand{\cH}{\mathcal{H}}
\newcommand{\cI}{\mathcal{I}}
\newcommand{\cJ}{\mathcal{J}}
\newcommand{\cK}{\mathcal{K}}
\newcommand{\cL}{\mathcal{L}}
\newcommand{\cM}{\mathcal{M}}
\newcommand{\cN}{\mathcal{N}}
\newcommand{\cO}{\mathcal{O}}
\newcommand{\cP}{\mathcal{P}}
\newcommand{\cQ}{\mathcal{Q}}
\newcommand{\cR}{\mathcal{R}}
\newcommand{\cS}{\mathcal{S}}
\newcommand{\cT}{\mathcal{T}}
\newcommand{\cU}{\mathcal{U}}
\newcommand{\cV}{\mathcal{V}}
\newcommand{\cW}{\mathcal{W}}
\newcommand{\cX}{\mathcal{X}}
\newcommand{\cY}{\mathcal{Y}}
\newcommand{\cZ}{\mathcal{Z}}


%%%%%%%%%%%%% mathematical fraktur  %%%%%%%%%%%%%%%%%%%%%
\newcommand{\mf}[1]{\mathfrak{#1}}

\newcommand{\fA}{\mathfrak{A}}
\newcommand{\fB}{\mathfrak{B}}
\newcommand{\fC}{\mathfrak{C}}
\newcommand{\fD}{\mathfrak{D}}
\newcommand{\fE}{\mathfrak{E}}
\newcommand{\fF}{\mathfrak{F}}
\newcommand{\fG}{\mathfrak{G}}
\newcommand{\fH}{\mathfrak{H}}
\newcommand{\fI}{\mathfrak{I}}
\newcommand{\fJ}{\mathfrak{J}}
\newcommand{\fK}{\mathfrak{K}}
\newcommand{\fL}{\mathfrak{L}}
\newcommand{\fM}{\mathfrak{M}}
\newcommand{\fN}{\mathfrak{N}}
\newcommand{\fO}{\mathfrak{O}}
\newcommand{\fP}{\mathfrak{P}}
\newcommand{\fQ}{\mathfrak{Q}}
\newcommand{\fR}{\mathfrak{R}}
\newcommand{\fS}{\mathfrak{S}}
\newcommand{\fT}{\mathfrak{T}}
\newcommand{\fU}{\mathfrak{U}}
\newcommand{\fV}{\mathfrak{V}}
\newcommand{\fW}{\mathfrak{W}}
\newcommand{\fX}{\mathfrak{X}}
\newcommand{\fY}{\mathfrak{Y}}
\newcommand{\fZ}{\mathfrak{Z}}


%%%%%%%%%%%%% mathematical script (euler)  %%%%%%%%%%%%%%%%%%%%%
\newcommand{\ms}[1]{\mathscr{#1}}

\newcommand{\sA}{\mathscr{A}}
\newcommand{\sB}{\mathscr{B}}
\newcommand{\sC}{\mathscr{C}}
\newcommand{\sD}{\mathscr{D}}
\newcommand{\sE}{\mathscr{E}}
\newcommand{\sF}{\mathscr{F}}
\newcommand{\sG}{\mathscr{G}}
\newcommand{\sH}{\mathscr{H}}
\newcommand{\sI}{\mathscr{I}}
\newcommand{\sJ}{\mathscr{J}}
\newcommand{\sK}{\mathscr{K}}
\newcommand{\sL}{\mathscr{L}}
\newcommand{\sM}{\mathscr{M}}
\newcommand{\sN}{\mathscr{N}}
\newcommand{\sO}{\mathscr{O}}
\newcommand{\sP}{\mathscr{P}}
\newcommand{\sQ}{\mathscr{Q}}
\newcommand{\sR}{\mathscr{R}}
\newcommand{\sS}{\mathscr{S}}
\newcommand{\sT}{\mathscr{T}}
\newcommand{\sU}{\mathscr{U}}
\newcommand{\sV}{\mathscr{V}}
\newcommand{\sW}{\mathscr{W}}
\newcommand{\sX}{\mathscr{X}}
\newcommand{\sY}{\mathscr{Y}}
\newcommand{\sZ}{\mathscr{Z}}


%%%%%%%%%%    Math operators    %%%%%%%%%%%%%%%%%%%%%%%%%%%

\renewcommand{\Re}{\mathop{\textnormal{Re}}}  % real part
\renewcommand{\Im}{\mathop{\textnormal{Im}}}  % imaginary part


%%%%%%%%%%%  FURTHER COMMANDS  %%%%%%%%%%%%%%%

\newcommand{\Id}{\mathrm{Id}}


%%%%%%%%%%%  STUDENT COMMANDS  %%%%%%%%%%%%%%%
%% Hier können Sie Ihre eigene LaTeX kommandos hinzufügen. %%
\newtheorem*{theorem*}{Theorem}
%% this allows for theorems which are not automatically numbered
\newtheorem{definition}{Definition}
\newtheorem{theorem}{Theorem}
\newtheorem{lemma}{Lemma}
\newtheorem{example}{Example}

%%%%%%%%%%%%%%%%%%%%%%%%%%%%%%%%%%%%%%%%%%%


\begin{document}
{\Large\bf }\\  tags:\  {hirsch}\\
\hrule
%%%%%%%%%%%%%%%%%%%%%%%%%%%%%%%%%%%%%%%
\bigskip
Sei $f:E\to\mathbb{R} ^{n}$ stetig und $E$ kompakt, so ist $f$ gleichmäßig stetig.
\begin{proof}
	Beweis durch Widerspruch.\\
	Sei $f$ nicht gleichmäßig stetig.\\
	$\exists\varepsilon>0: \forall\delta>0: \exists x_{\delta},y_{\delta}\in D:|x_{\delta}-y_\delta|<\delta,\text{aber }|f(x_{\delta})-f(y_{\delta})|\geq\varepsilon$\medskip

	Insbesondere:\medskip

	Sei $\delta = \frac{1}{n}$\medskip

	Es gibt dann ein $x_{n},y_{n}$, sodass:$$
		|x_{n}-y_{n}|<\frac{1}{n}, \textnormal{aber } |f(x_{n})-f(y_{n})|\geq \varepsilon
	$$
	$(x_{n})_{n}$ stellt dabei eine Folge in der kompakten Menge $E$ dar.\\
	Es gibt also eine Teilfolge $(x_{n_{k}})_{k}$ von $(x_{n})_{n}$ mit $\lim \limits_{k \to\infty}x_{n_{k}}=x_{\infty }$\\
	\begin{align*}
		|y_{n_{k}}-x_{\infty }|\leq |x_{n_{k}}-x_{\infty }|+|x_{n_{k}}-y_{n_{k}}|\leq |x_{n_{k}}-x_{\infty }|+\frac{1}{n_{k}}\to 0
	\end{align*}
	Somit konvergiert auch die Teilfolge $(y_{n_{k}})_{k}$ gegen $x$.\bigskip

	Da $f$ stetig ist, folgt aus dem Folgenkriterium:
	\begin{align*}
		 & f(x_{n_{k}})\to f(x)                                                                                                           \\
		 & f(y_{n_{k}})\to f(x)                                                                                                           \\
		 & \textnormal{Die folgen $f(y_{n_{k}})$ und $f(x_{n_{k}})$ sind also äquivalent.}                                                \\
		 & \lim \limits_{k \to\infty}|f(x_{n_{k}})-f(y_{n_{k}})|\leq \lim \limits_{k \to\infty} |f(x_{n_{k}})-f(x)|+|f(y_{n_{k}})-f(x)|=0
	\end{align*}
	$\forall \varepsilon >0:|f(x_{n_{k}})-f(y_{n_{k}})|<\varepsilon, \forall k>K$
	\bigskip


	Es gibt also ein $\delta = \frac{1}{n_{k}}$, für das gilt:\\ $$|x-y|<\delta \textnormal{ (Denn so sind $x_{n_{k}}$ und $y_{n_{k}}$ definiert}) \implies f(x_{n_{k}})-f(y_{n_{k}})<\varepsilon, \forall  \varepsilon >0$$.\\
	Dies ist ein Widerpsruch zu der Annahme, das $f$ nicht gleichmäßig stetig ist.
\end{proof}
\end{document}

