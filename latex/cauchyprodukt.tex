
\documentclass[a4paper,oneside,11pt]{article}
\usepackage[utf8]{inputenc}

\usepackage[ngerman]{babel} % For correct hyphenation

\usepackage{mathtools}
\usepackage{amssymb,amsmath,amsthm}
\usepackage[mathscr]{eucal}

\usepackage[textwidth=17cm,top=1.5cm,bottom=1.5cm,nohead]{geometry}

\setlength{\parindent}{0mm} % no paragraph indentation


%%%%%%%%%%%%%%%%%%%%%%%%%%%%%%%%%%%%%%%%%%%%%%%%%%%%%%%

% Abbreviations

%%%%%%%%%%%%%%%%%%%%%%%%%%%%%%%%%%%%%%%%%%%%%%%%%%%%%%%
% single letters in different fonts 

%%%%%%%%%%%% mathematical bold  %%%%%%%%%%%%%%%%%%%%

\newcommand{\bA}{\mathbb{A}}
\newcommand{\bB}{\mathbb{B}}
\newcommand{\bC}{\mathbb{C}}
\newcommand{\bD}{\mathbb{D}}
\newcommand{\bE}{\mathbb{E}}
\newcommand{\bF}{\mathbb{F}}
\newcommand{\bG}{\mathbb{G}}
\newcommand{\bH}{\mathbb{H}}
\newcommand{\bI}{\mathbb{I}}
\newcommand{\bJ}{\mathbb{J}}
\newcommand{\bK}{\mathbb{K}}
\newcommand{\bL}{\mathbb{L}}
\newcommand{\bM}{\mathbb{M}}
\newcommand{\bN}{\mathbb{N}}
\newcommand{\bO}{\mathbb{O}}
\newcommand{\bP}{\mathbb{P}}
\newcommand{\bQ}{\mathbb{Q}}
\newcommand{\bR}{\mathbb{R}}
\newcommand{\bS}{\mathbb{S}}
\newcommand{\bT}{\mathbb{T}}
\newcommand{\bU}{\mathbb{U}}
\newcommand{\bV}{\mathbb{V}}
\newcommand{\bW}{\mathbb{W}}
\newcommand{\bX}{\mathbb{X}}
\newcommand{\bY}{\mathbb{Y}}
\newcommand{\bZ}{\mathbb{Z}}


%%%%%%%%% calligraphic %%%%%%%%%%%%%%%%%%%%%%%
\newcommand{\mc}[1]{\mathcal{#1}}

\newcommand{\cA}{\mathcal{A}}
\newcommand{\cB}{\mathcal{B}}
\newcommand{\cC}{\mathcal{C}}
\newcommand{\cD}{\mathcal{D}}
\newcommand{\cE}{\mathcal{E}}
\newcommand{\cF}{\mathcal{F}}
\newcommand{\cG}{\mathcal{G}}
\newcommand{\cH}{\mathcal{H}}
\newcommand{\cI}{\mathcal{I}}
\newcommand{\cJ}{\mathcal{J}}
\newcommand{\cK}{\mathcal{K}}
\newcommand{\cL}{\mathcal{L}}
\newcommand{\cM}{\mathcal{M}}
\newcommand{\cN}{\mathcal{N}}
\newcommand{\cO}{\mathcal{O}}
\newcommand{\cP}{\mathcal{P}}
\newcommand{\cQ}{\mathcal{Q}}
\newcommand{\cR}{\mathcal{R}}
\newcommand{\cS}{\mathcal{S}}
\newcommand{\cT}{\mathcal{T}}
\newcommand{\cU}{\mathcal{U}}
\newcommand{\cV}{\mathcal{V}}
\newcommand{\cW}{\mathcal{W}}
\newcommand{\cX}{\mathcal{X}}
\newcommand{\cY}{\mathcal{Y}}
\newcommand{\cZ}{\mathcal{Z}}


%%%%%%%%%%%%% mathematical fraktur  %%%%%%%%%%%%%%%%%%%%%
\newcommand{\mf}[1]{\mathfrak{#1}}

\newcommand{\fA}{\mathfrak{A}}
\newcommand{\fB}{\mathfrak{B}}
\newcommand{\fC}{\mathfrak{C}}
\newcommand{\fD}{\mathfrak{D}}
\newcommand{\fE}{\mathfrak{E}}
\newcommand{\fF}{\mathfrak{F}}
\newcommand{\fG}{\mathfrak{G}}
\newcommand{\fH}{\mathfrak{H}}
\newcommand{\fI}{\mathfrak{I}}
\newcommand{\fJ}{\mathfrak{J}}
\newcommand{\fK}{\mathfrak{K}}
\newcommand{\fL}{\mathfrak{L}}
\newcommand{\fM}{\mathfrak{M}}
\newcommand{\fN}{\mathfrak{N}}
\newcommand{\fO}{\mathfrak{O}}
\newcommand{\fP}{\mathfrak{P}}
\newcommand{\fQ}{\mathfrak{Q}}
\newcommand{\fR}{\mathfrak{R}}
\newcommand{\fS}{\mathfrak{S}}
\newcommand{\fT}{\mathfrak{T}}
\newcommand{\fU}{\mathfrak{U}}
\newcommand{\fV}{\mathfrak{V}}
\newcommand{\fW}{\mathfrak{W}}
\newcommand{\fX}{\mathfrak{X}}
\newcommand{\fY}{\mathfrak{Y}}
\newcommand{\fZ}{\mathfrak{Z}}


%%%%%%%%%%%%% mathematical script (euler)  %%%%%%%%%%%%%%%%%%%%%
\newcommand{\ms}[1]{\mathscr{#1}}

\newcommand{\sA}{\mathscr{A}}
\newcommand{\sB}{\mathscr{B}}
\newcommand{\sC}{\mathscr{C}}
\newcommand{\sD}{\mathscr{D}}
\newcommand{\sE}{\mathscr{E}}
\newcommand{\sF}{\mathscr{F}}
\newcommand{\sG}{\mathscr{G}}
\newcommand{\sH}{\mathscr{H}}
\newcommand{\sI}{\mathscr{I}}
\newcommand{\sJ}{\mathscr{J}}
\newcommand{\sK}{\mathscr{K}}
\newcommand{\sL}{\mathscr{L}}
\newcommand{\sM}{\mathscr{M}}
\newcommand{\sN}{\mathscr{N}}
\newcommand{\sO}{\mathscr{O}}
\newcommand{\sP}{\mathscr{P}}
\newcommand{\sQ}{\mathscr{Q}}
\newcommand{\sR}{\mathscr{R}}
\newcommand{\sS}{\mathscr{S}}
\newcommand{\sT}{\mathscr{T}}
\newcommand{\sU}{\mathscr{U}}
\newcommand{\sV}{\mathscr{V}}
\newcommand{\sW}{\mathscr{W}}
\newcommand{\sX}{\mathscr{X}}
\newcommand{\sY}{\mathscr{Y}}
\newcommand{\sZ}{\mathscr{Z}}


%%%%%%%%%%    Math operators    %%%%%%%%%%%%%%%%%%%%%%%%%%%

\renewcommand{\Re}{\mathop{\textnormal{Re}}}  % real part
\renewcommand{\Im}{\mathop{\textnormal{Im}}}  % imaginary part


%%%%%%%%%%%  FURTHER COMMANDS  %%%%%%%%%%%%%%%

\newcommand{\Id}{\mathrm{Id}}


%%%%%%%%%%%  STUDENT COMMANDS  %%%%%%%%%%%%%%%
%% Hier können Sie Ihre eigene LaTeX kommandos hinzufügen. %%
\newtheorem*{theorem*}{Theorem}
%% this allows for theorems which are not automatically numbered
\newtheorem{definition}{Definition}
\newtheorem{theorem}{Theorem}
\newtheorem{lemma}{Lemma}
\newtheorem{example}{Example}

%%%%%%%%%%%%%%%%%%%%%%%%%%%%%%%%%%%%%%%%%%%


\begin{document}
{\Large\bf Cauchy-Produkt}\\  tags:\  {Hirsch, Cauchy-Produkt}\\
\hrule
%%%%%%%%%%%%%%%%%%%%%%%%%%%%%%%%%%%%%%%
\subsection*{I}
Es kann eine Abzählung gefunden werden $\phi: \mathbb{N} \to\mathbb{N} \times\mathbb{N}$,\\
sodass $(N+1)^2$ natürliche Zahlen diese Tupel $(a,b)\in\mathbb{N} \times\mathbb{N}$ abzählen: \\
$$\{(i,j): \max \{i,j\} \leq N^2\}$$\\

\mediumskip
$\phi(k)=(\phi_1(k), \phi_2(k))\in \mathbb{N} \times\mathbb{N}$


\bigskip
Sei die Elemnte der Tupel werden miteinander multiplizier und die Produkte summiert.\\
Es ergeben sich somit $(N+1)^2$ Produkte,da es genau so viele Tupel gibt.

\bigskip

Die Summe der Produkte dieser Tupel kann auch so geschrieben werden:
$$
\left(\sum_{i=0}^{N^2} a_i\right)\left(\sum_{i=0}^{N^2} b_i\right)
$$
\\
$$
\sum^{(N+1)^2}_{i=0} a_i \cdot b_i = \left(\sum_{i=0}^{N^2} a_i\right)\left(\sum_{i=0}^{N^2} b_i\right)
$$
$$
\sum^{\infty}_{i=0} a_i \cdot b_i = \left(\sum_{i=0}^{\infty} a_i\right)\left(\sum_{i=0}^{\infty} b_i\right)
$$
\subsection*{II}
\bigskip
Es kann eine Abzählung gefunden werden $\psi: \mathbb{N} \to\mathbb{N} \times\mathbb{N}$,\\
sodass $\frac{(N+1)(N+2)}{2}$ natürliche Zahlen diese Tupel $(a,b)\in\mathbb{N} \times\mathbb{N}$ abzählen: \\
$$\{(i,j-i)|0 \leq i \leq N, 0 \leq j \leq i, i+j \leq N\}$$\\

\mediumskip
$\psi(k)=(\psi_1(k), \psi_2(k))\in \mathbb{N} \times\mathbb{N}$


\bigskip
Sei die Elemnte der Tupel werden miteinander multiplizier und die Produkte summiert.\\
Es ergeben sich somit $\frac{(N+1)(N+2)}{2}$ Produkte,da es genau so viele Tupel gibt.

\bigskip

Die Summe der Produkte dieser Tupel kann auch so geschrieben werden:
$$
\sum_{i=0}^{N}\sum^i_{j=0} a_{i-j}b_j
$$



\bigskip

Da es aber eine Abzählung dieser Tupel auf $\frac{(N+1)(N+2)}{2}$ natürliche Zahlen gibt, kann für jeder diser Tupel $(n,m)$, eine natürliche zahl $k \leq \frac{(N+1)(N+2)}{2}$ gefunden werden, sodass $$\psi(k)=(n,m)=(\psi_1(k), \psi_2(k))$$
Somit gilt:
$$
a_n\cdot b_m= a_{\psi_1(k)}\cdot  b_{\psi_2(k)}
$$


\bigskip
Somit gibt es eine Abzählung, sodass $$
\sum_{i=0}^{N}\sum^i_{j=0} a_{i-j}b_j=\sum_{i=0}^{\frac{(N+1)(N+2)}{2}}a_{\psi_1(i)}b_{\psi_2{(i)}}
$$

$$
\sum_{i=0}^{\infty}\sum^i_{j=0} a_{i-j}b_j=\sum_{i=0}^{\infty}a_{\psi_1(i)}b_{\psi_2{(i)}}
$$
\section*{Fazit}
Wie gerade gezeigt,  gibt es zwei Abzählungen $\phi: \mathbb{N} \to \mathbb{N} \times \mathbb{N}$, $\psi: \mathbb{N} \to \mathbb{N} \times \mathbb{N}$.\\
$\tau=\psi^{-1}\circ \phi:\mathbb{N} \to\mathbb{N} $ ist eine Umordnung von $\mathbb{N}$.
\\

$$\sum^{\infty}_{i=0} a_i \cdot b_i = \left(\sum_{i=0}^{\infty} a_i\right)\left(\sum_{i=0}^{\infty} b_i\right)
$$
\end{document}

