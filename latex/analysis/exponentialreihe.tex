\documentclass[a4paper,oneside,11pt]{article}
\usepackage[utf8]{inputenc}

\usepackage[ngerman]{babel} % For correct hyphenation

\usepackage{mathtools}
\usepackage{amssymb,amsmath,amsthm}
\usepackage[mathscr]{eucal}

\usepackage[textwidth=17cm,top=1.5cm,bottom=1.5cm,nohead]{geometry}

\setlength{\parindent}{0mm} % no paragraph indentation


%%%%%%%%%%%%%%%%%%%%%%%%%%%%%%%%%%%%%%%%%%%%%%%%%%%%%%%

% Abbreviations

%%%%%%%%%%%%%%%%%%%%%%%%%%%%%%%%%%%%%%%%%%%%%%%%%%%%%%%
% single letters in different fonts 

%%%%%%%%%%%% mathematical bold  %%%%%%%%%%%%%%%%%%%%

\newcommand{\bA}{\mathbb{A}}
\newcommand{\bB}{\mathbb{B}}
\newcommand{\bC}{\mathbb{C}}
\newcommand{\bD}{\mathbb{D}}
\newcommand{\bE}{\mathbb{E}}
\newcommand{\bF}{\mathbb{F}}
\newcommand{\bG}{\mathbb{G}}
\newcommand{\bH}{\mathbb{H}}
\newcommand{\bI}{\mathbb{I}}
\newcommand{\bJ}{\mathbb{J}}
\newcommand{\bK}{\mathbb{K}}
\newcommand{\bL}{\mathbb{L}}
\newcommand{\bM}{\mathbb{M}}
\newcommand{\bN}{\mathbb{N}}
\newcommand{\bO}{\mathbb{O}}
\newcommand{\bP}{\mathbb{P}}
\newcommand{\bQ}{\mathbb{Q}}
\newcommand{\bR}{\mathbb{R}}
\newcommand{\bS}{\mathbb{S}}
\newcommand{\bT}{\mathbb{T}}
\newcommand{\bU}{\mathbb{U}}
\newcommand{\bV}{\mathbb{V}}
\newcommand{\bW}{\mathbb{W}}
\newcommand{\bX}{\mathbb{X}}
\newcommand{\bY}{\mathbb{Y}}
\newcommand{\bZ}{\mathbb{Z}}


%%%%%%%%% calligraphic %%%%%%%%%%%%%%%%%%%%%%%
\newcommand{\mc}[1]{\mathcal{#1}}

\newcommand{\cA}{\mathcal{A}}
\newcommand{\cB}{\mathcal{B}}
\newcommand{\cC}{\mathcal{C}}
\newcommand{\cD}{\mathcal{D}}
\newcommand{\cE}{\mathcal{E}}
\newcommand{\cF}{\mathcal{F}}
\newcommand{\cG}{\mathcal{G}}
\newcommand{\cH}{\mathcal{H}}
\newcommand{\cI}{\mathcal{I}}
\newcommand{\cJ}{\mathcal{J}}
\newcommand{\cK}{\mathcal{K}}
\newcommand{\cL}{\mathcal{L}}
\newcommand{\cM}{\mathcal{M}}
\newcommand{\cN}{\mathcal{N}}
\newcommand{\cO}{\mathcal{O}}
\newcommand{\cP}{\mathcal{P}}
\newcommand{\cQ}{\mathcal{Q}}
\newcommand{\cR}{\mathcal{R}}
\newcommand{\cS}{\mathcal{S}}
\newcommand{\cT}{\mathcal{T}}
\newcommand{\cU}{\mathcal{U}}
\newcommand{\cV}{\mathcal{V}}
\newcommand{\cW}{\mathcal{W}}
\newcommand{\cX}{\mathcal{X}}
\newcommand{\cY}{\mathcal{Y}}
\newcommand{\cZ}{\mathcal{Z}}


%%%%%%%%%%%%% mathematical fraktur  %%%%%%%%%%%%%%%%%%%%%
\newcommand{\mf}[1]{\mathfrak{#1}}

\newcommand{\fA}{\mathfrak{A}}
\newcommand{\fB}{\mathfrak{B}}
\newcommand{\fC}{\mathfrak{C}}
\newcommand{\fD}{\mathfrak{D}}
\newcommand{\fE}{\mathfrak{E}}
\newcommand{\fF}{\mathfrak{F}}
\newcommand{\fG}{\mathfrak{G}}
\newcommand{\fH}{\mathfrak{H}}
\newcommand{\fI}{\mathfrak{I}}
\newcommand{\fJ}{\mathfrak{J}}
\newcommand{\fK}{\mathfrak{K}}
\newcommand{\fL}{\mathfrak{L}}
\newcommand{\fM}{\mathfrak{M}}
\newcommand{\fN}{\mathfrak{N}}
\newcommand{\fO}{\mathfrak{O}}
\newcommand{\fP}{\mathfrak{P}}
\newcommand{\fQ}{\mathfrak{Q}}
\newcommand{\fR}{\mathfrak{R}}
\newcommand{\fS}{\mathfrak{S}}
\newcommand{\fT}{\mathfrak{T}}
\newcommand{\fU}{\mathfrak{U}}
\newcommand{\fV}{\mathfrak{V}}
\newcommand{\fW}{\mathfrak{W}}
\newcommand{\fX}{\mathfrak{X}}
\newcommand{\fY}{\mathfrak{Y}}
\newcommand{\fZ}{\mathfrak{Z}}


%%%%%%%%%%%%% mathematical script (euler)  %%%%%%%%%%%%%%%%%%%%%
\newcommand{\ms}[1]{\mathscr{#1}}

\newcommand{\sA}{\mathscr{A}}
\newcommand{\sB}{\mathscr{B}}
\newcommand{\sC}{\mathscr{C}}
\newcommand{\sD}{\mathscr{D}}
\newcommand{\sE}{\mathscr{E}}
\newcommand{\sF}{\mathscr{F}}
\newcommand{\sG}{\mathscr{G}}
\newcommand{\sH}{\mathscr{H}}
\newcommand{\sI}{\mathscr{I}}
\newcommand{\sJ}{\mathscr{J}}
\newcommand{\sK}{\mathscr{K}}
\newcommand{\sL}{\mathscr{L}}
\newcommand{\sM}{\mathscr{M}}
\newcommand{\sN}{\mathscr{N}}
\newcommand{\sO}{\mathscr{O}}
\newcommand{\sP}{\mathscr{P}}
\newcommand{\sQ}{\mathscr{Q}}
\newcommand{\sR}{\mathscr{R}}
\newcommand{\sS}{\mathscr{S}}
\newcommand{\sT}{\mathscr{T}}
\newcommand{\sU}{\mathscr{U}}
\newcommand{\sV}{\mathscr{V}}
\newcommand{\sW}{\mathscr{W}}
\newcommand{\sX}{\mathscr{X}}
\newcommand{\sY}{\mathscr{Y}}
\newcommand{\sZ}{\mathscr{Z}}


%%%%%%%%%%    Math operators    %%%%%%%%%%%%%%%%%%%%%%%%%%%

\renewcommand{\Re}{\mathop{\textnormal{Re}}}  % real part
\renewcommand{\Im}{\mathop{\textnormal{Im}}}  % imaginary part


%%%%%%%%%%%  FURTHER COMMANDS  %%%%%%%%%%%%%%%

\newcommand{\Id}{\mathrm{Id}}


%%%%%%%%%%%  STUDENT COMMANDS  %%%%%%%%%%%%%%%
%% Hier können Sie Ihre eigene LaTeX kommandos hinzufügen. %%
\newtheorem*{theorem*}{Theorem}
%% this allows for theorems which are not automatically numbered
\newtheorem{definition}{Definition}
\newtheorem{theorem}{Theorem}
\newtheorem{lemma}{Lemma}
\newtheorem{example}{Example}

%%%%%%%%%%%%%%%%%%%%%%%%%%%%%%%%%%%%%%%%%%%


\begin{document}
{\Large\bf Exponentialreihe}\\  tags:\  {hirsch}\\
\hrule
%%%%%%%%%%%%%%%%%%%%%%%%%%%%%%%%%%%%%%%
\bigskip
\begin{theorem*}Satz 8.1.1
	Es gibt genau eine Funktion $exp:\mathbb{C} \to \mathbb{C} $ mit  den Folgenden Eigenschaften:
	\begin{itemize}
		\item (A) $\exp(z+w)=\exp(z)\cdot \exp(w)$ ("Additionseigenschaft")
		\item (W) $\lim \limits_{z \to 0}\frac{\exp(z)-1}{z}=1$ ("Wachstum")
	\end{itemize}
	$(\exp(z)=\sum_{n}^{\infty}\frac{z^{k}}{n!})$\bigskip


	Für diese Funktion gilt:
	\begin{itemize}
		\item (i) $\exp(z)=\sum_{n=0}^{\infty}\frac{z^{n}}{n!}$
		\item (ii) $\exp(z)=\lim \limits_{n \to\infty}(1+\frac{z}{n})^{n}$
		\item (iii) $\exp(z)$ ist stetig und $e^{q}=\exp(q), \forall q\in \mathbb{Q}$, wobei $e=\exp(1)$ ist die Eulersche Zahl
	\end{itemize}
\end{theorem*}
Sei eine Funktion $f:\mathbb{C} \to \mathbb{C} $, die die Eigenschaften $(A),(W)$ erfüllt.
\section*{1.}
\textbf{1.1}
$f(z)=f(n\cdot \frac{z}{n})\overset{(A)}{=}f(\frac{z}{n})^{n}$
\bigskip


\textbf{1.2}\\
Setze $z_{n}:=n(f(\frac{z}{n})-1)$\\
Es gilt: \\
\begin{align}
	\lim \limits_{n \to\infty}z_{n}=  n(f(\frac{z}{n})-1)= & n\cdot (\frac{z}{n})\cdot \underbrace{\left(\frac{f(\frac{z}{n})-1}{\frac{z}{n}}\right)}_{\textnormal{gegen $1$, siehe eigenschaft $W$ für $\exp(\frac{z}{n})$}} \textnormal{\textit{(Erweiterung mit  $\frac{z}{n}$)}} \\
	=                                                      & z\cdot 1=z
\end{align}
\textbf{1.3}
$$
	f\left(\frac{z}{n}\right)^{n}=\left(1+\frac{n(f(\frac{z}{n})-1)}{n}\right)^{n} \textnormal{(\textit{Erweiterung})}
$$
\section*{Lemma 8.2: Fundamentallemma der Exponentialfunktion}
\begin{lemma}
	Für alle $(z_{n})_{n\in \mathbb{N} }$ mit $\lim \limits_{n \to\infty}z_{n}=z$ gilt:
	\begin{align}
		\lim \limits_{n \to\infty}(1+\frac{z_{n}}{n})^{n}=\sum_{k=0}^{\infty}\frac{z^{k}}{k!}
	\end{align}
\end{lemma}

\begin{proof}\bigskip

	Einzigartigkeit:
	\begin{align}
		f(z)\overset{1.1}{=}\lim \limits_{n \to\infty}f(\frac{z}{n})^{n}  \overset{1.3}{=} & \lim \limits_{n \to\infty}  \left(1+\frac{n(f(\frac{z}{n})-1)}{\frac{z}{n}}\right)^{n} \\
		\overset{8.2}{=}                                                                   & \sum_{k=0}^{\infty}\frac{z^{k}}{k!}=\exp(z)
	\end{align}
	$ii)$\bigskip
	\begin{align}
		\exp(z)=\sum_{k=0}^{\infty}\frac{z^{k}}{k!}\overset{8.2}{=} & \lim \limits_{n \to\infty}(1+\frac{z_{n}}{n})^{n}             \\
		                                                            & \overset{1.2}{=}\lim \limits_{n \to\infty}(1+\frac{z}{n})^{n}
	\end{align}

	$iii)$


\end{proof}
\end{document}

